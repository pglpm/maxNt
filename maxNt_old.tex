\documentclass{article}

% if you need to pass options to natbib, use, e.g.:
% \PassOptionsToPackage{numbers, compress}{natbib}
% before loading nips_2017
%
% to avoid loading the natbib package, add option nonatbib:
% \usepackage[nonatbib]{nips_2017}

\usepackage[final,nonatbib]{nips_2017}

% to compile a camera-ready version, add the [final] option, e.g.:
% \usepackage[final]{nips_2017}

\usepackage[T1]{fontenc}    % use 8-bit T1 fonts
\usepackage[utf8]{inputenc} % allow utf-8 input
%\usepackage{hyperref}       % hyperlinks
%\usepackage{url}            % simple URL typesetting
\usepackage{booktabs}       % professional-quality tables
%\usepackage{amsfonts}       % blackboard math symbols
\usepackage{nicefrac}       % compact symbols for 1/2, etc.
\usepackage{microtype}      % microtypography
\usepackage{textcomp}
\usepackage{amsmath}

\usepackage{mathtools}
\setlength{\multlinegap}{0pt}
% \newlength{\Normalbaselineskip}
% \setlength{\Normalbaselineskip}{\baselineskip}
% \appto\MultlinedHook{\setlength\baselineskip{\Normalbaselineskip}}

\usepackage{amssymb}
\usepackage{amsxtra}
\usepackage{amsthm}
\usepackage{bm}

\usepackage[british]{babel}
\selectlanguage{british}
% \newcommand*{\langfrench}{\foreignlanguage{french}}
% \newcommand*{\langgerman}{\foreignlanguage{german}}
% \newcommand*{\langitalian}{\foreignlanguage{italian}}
% \newcommand*{\langswedish}{\foreignlanguage{swedish}}
% \newcommand*{\langlatin}{\foreignlanguage{latin}}
% \newcommand*{\langnohyph}{\foreignlanguage{nohyphenation}}
% \newcommand*{\latin}[1]{\langlatin{#1}}
% \newcommand*{\french}[1]{\emph{\langfrench{#1}}}
% \newcommand*{\german}[1]{\emph{\langgerman{#1}}}

\usepackage[autostyle=false,autopunct=false,english=british]{csquotes}
\setquotestyle{american}
\iffalse
\usepackage[shortlabels,inline]{enumitem}
% to replace paralist environments with enumitem ones:
\SetEnumitemKey{para}{itemindent=\parindent,leftmargin=0pt,listparindent=\parindent,parsep=0pt,itemsep=\topsep}
\setlist[enumerate,2]{label=\alph*.}
\setlist[enumerate]{leftmargin=\parindent}
\setlist[itemize]{leftmargin=\parindent}
\setlist[description]{leftmargin=\parindent}
\fi

\usepackage[backend=biber,mcite,subentry,citestyle=numeric-comp,bibstyle=numericbringhurst,autopunct=false,sorting=none,sortcites=false,natbib=false,maxnames=8,minnames=8,giveninits=true,block=space,hyperref=true,defernumbers=false,useprefix=true,language=british]{biblatex}
\renewcommand*{\finalnamedelim}{, }
\setcounter{biburlnumpenalty}{1}
\setcounter{biburlucpenalty}{0}
\setcounter{biburllcpenalty}{1}

\usepackage{graphicx}

\usepackage[usenames]{xcolor}
\definecolor{myblue}{RGB}{51,34,136}
\definecolor{mygreen}{RGB}{17,119,51}
\definecolor{myred}{RGB}{136,34,85}
\definecolor{myyellow}{RGB}{153,153,51}
\definecolor{mylightyellow}{RGB}{221,204,119}
\newcommand*{\mynote}[1]{ {\color[RGB]{68,170,153}\maltese\ #1}}

\usepackage{hyperref}
\providecommand{\href}[2]{#2}
\providecommand{\eprint}[2]{\texttt{\href{#1}{#2}}}
\providecommand*{\urlalt}{\href}
\hypersetup{
colorlinks=true,%
bookmarksnumbered,%
pdfborder={0 0 0.25},%
citebordercolor={0.2 0.1333 0.5333},%bluish
citecolor=myblue,%
linkbordercolor={0.0667 0.4667 0.2},%greenish
linkcolor=myred,%
urlbordercolor={0.5333 0.1333 0.3333},%reddish
urlcolor=mygreen,%
breaklinks=true,%
pdftitle = {Hidden network-size assumptions in the maximum-entropy method},%***
pdfauthor = {Rostami, Porta Mana, Torre}%***
}
\usepackage[depth=4]{bookmark}
\usepackage{mathdots}


%%%%%%%%%%%%%% macros
\newcommand*{\widebar}[1]{{\mkern 1.5mu\skew{2}\overline{\mkern-1.5mu#1\mkern-1.5mu}\mkern 1.5mu}}
\newcommand*{\av}{\overline} %pop average
\newcommand*{\sav}{\widehat} %subpop average
\newcommand*{\yxx}{x}
\newcommand*{\yx}{\bm{\yxx}}%subpop state
\newcommand*{\yxs}{\sav{\yx}}%subpop av state
\newcommand*{\yxxs}{\sav{\yx\yx}}
\newcommand*{\yX}{\bm{X}}%pop state
\newcommand*{\yXf}{\av{\yX}}%pop av state
\newcommand*{\yr}{\bm{r}}%subpop value
\newcommand*{\yrs}{\sav{\yr}}%subpop av value
\newcommand*{\yR}{\bm{R}}%pop value
\newcommand*{\yRf}{\av{\yR}}%pop av value
\newcommand*{\yN}{N}
\newcommand*{\yM}{M}
\newcommand*{\yee}{L}

\newcommand*{\mee}{maximum entropy}
\newcommand*{\me}{maximum-entropy}
\title{Hidden network-size assumptions\\in the maximum-entropy method}

% The \author macro works with any number of authors. There are two
% commands used to separate the names and addresses of multiple
% authors: \And and \AND.
%
% Using \And between authors leaves it to LaTeX to determine where to
% break the lines. Using \AND forces a line break at that point. So,
% if LaTeX puts 3 of 4 authors names on the first line, and the last
% on the second line, try using \AND instead of \And before the third
% author name.

\author{ V. Rostami\\
  Forschungszentrum Jülich INM-6\\
  \texttt{v.rostami@fz-juelich.de}
  \And
  P.G.L. Porta Mana\\who knows
  \And E. Torre\\Chocolateland
  %
% David S.~Hippocampus\thanks{Use footnote for providing further information
%   about author (webpage, alternative
%   address)---\emph{not} for acknowledging funding agencies.} \\
% Department of Computer Science\\
% Cranberry-Lemon University\\
% Pittsburgh, PA 15213 \\
% \texttt{hippo@cs.cranberry-lemon.edu} \\
  %% examples of more authors
  %% \And
  %% Coauthor \\
  %% Affiliation \\
  %% Address \\
  %% \texttt{email} \\
  %% \AND
  %% Coauthor \\
  %% Affiliation \\
  %% Address \\
  %% \texttt{email} \\
  %% \And
  %% Coauthor \\
  %% Affiliation \\
  %% Address \\
  %% \texttt{email} \\
  %% \And
  %% Coauthor \\
  %% Affiliation \\
  %% Address \\
  %% \texttt{email} \\
}

\begin{document}
% \nipsfinalcopy is no longer used

\maketitle

\begin{abstract}
***%  The abstract must be limited to one paragraph.
\end{abstract}

\section{Implicit assumptions in the maximum-entropy method}

The \me\ method is been used in neuroscience blahblah***

In this note we would like to analyse and warn about a subtle assumption
behind the \me\ method when it is applied to a network. It can informally
be put this way:
\begin{quote}
  \emph{the \me\ method assumes that the network it is applied to is
    completely isolated from any larger network.}
\end{quote}

\mynote{Version 1}
The \me\ method does not construct a probability distribution out of
nothing, but starting from a uniform distribution. A uniform distribution
is an innocuous assumption for a set of non-composite events, like the
outcomes of a die roll, and also for some sets of composite events, like
the outcomes of the roll of two dice. In the latter case multiplicities
appear.

When applied to a subnetwork, the \me\ method assumes that the uniform over
the larger network is uniform, and therefore factorizable. The new
distribution of the subnetwork will not be uniform, but that of the full
network will still be factorizable into the one for the subnetwork and the
rest.

A uniform distribution, however, is not the right one when we suppose that
learning about an event may tell us something about a related event. For
example, consider 1\,000 tosses of a particular coin and assume a uniform
distribution over the possible $2^{1\,000}$ outcomes. If we learn that the
first 999 tosses yielded all \enquote{heads}, the probability calculus
tells us that the probability for the 1\,000th toss is still 50\%/50\%. It
is a consequence of our choice of a uniform distribution: we have
implicitly declared all tosses to be completely independent, completely
\emph{irrelevant} to one another. This fact is well-known in sampling
theory. A more telling example in fact is that of a presidential election
with two candidates: each citizen will vote for one or the other. We do
survey sampling on a large number of citizens to guess the election's
outcome. If we assumed a uniform distribution over the possible
combinations of choices of all citizens, our sampling would be completely
irrelevant for the choices of the rest of the population.




The latter example has many similarities with that of a neuronal binary
network. When we record the neuronal activity of a sample of neurons from a
brain area, we assume that our measurements can tell us something -- no
matter how vague or imprecise -- about the whole brain area. This means
that we are not assuming a uniform distribution over all possible states of
the area.




\mynote{Version 2}
This may come as a surprise. The method simply requires a number of
exhaustive and mutually exclusive events, and if these are composite events
the final distribution may have a multiplicity factor. When we consider the
$2^{\yN}$ states of $\yN$ units we are not excluding that these might be
marginals of $2^{\yM}$ states of $\yM$ units. Each one has the same
multiplicity $2^{\yM-\yN}$, but this constant multiplicity factor disappears
by normalization. So the method applies just as in the case of $\yN$ units
only, right?

Right, but 

Right, and that is where the problem lies. This way of counting of
multiplicities assumes an underlying 

Wrong. In our reasoning we have made subtle assumptions of independence
between the full network and the subnetwork. The problem is that the
counting of multiplicities is not based on simple enumeration, but already
involves probability considerations. Consider three cases with a full
population of two units, $\yM=2$, of which we consider one unit, $\yN=1$.
\begin{itemize}
\item First case: all four states are \emph{possible}. The two states of
  the first unit have multiplicity $2$ each. The usual \me\ distribution
  obtains.
\item Second case: only the states with at most one active unit are
  possible. The state $\yxx_1=1$ of the first unit has multiplicity
  $2$, and $\yxx_1=1$ has multiplicity $1$. The \me\ distribution has
  multiplicity factors.
\item Third case: states with at most one active unit are, say, $10^{9}$
  times more probable than the state with no active units. But all four
  states are \emph{possible}. By enumeration this case is like the first:
  multiplicities $(1,1)$. But by common sense it is more similar to the
  second: multiplicities $(2,1)$ for most practical purposes.
\end{itemize}

This simple example shows that the multiplicity inspection that must
precede a \me\ application already involves probability considerations at
the level of the full network. The usual reasoning by enumeration
implicitly assumes a uniform distribution or at least a \emph{factorizable}
distribution.

***If the distribution is factorizable, however, it means that examination of
the subnetwork \emph{cannot give us any insights about the network it is a
  part of}. This is obviously contrary to the reason why we made neuronal observations.

Consider the following ways of proceeding. We:
\begin{enumerate}
\item have a network with $\yN$ units, $2^{\yN}$ possible states
\item expect averages of $\yxs$ active neurons and $\yxxs$ active pairs
\item use maximum-entropy to choose a probability distribution for the
  states of the $\yN$ units conforming to our expectations.
\end{enumerate}

We:
\begin{enumerate}
\item have a network with $\yM$ units, $2^{\yM}$ possible states
\item expect that any subpopulation of $\yN$ units has $\yxs$ active
  neurons and $\yxxs$ active pairs
\item use maximum-entropy to choose a probability distribution for the
  states of the $\yM$ units conforming to our expectations
\item marginalize to find the probability distribution for the states of
  $\yN$ units.
\end{enumerate}


\subsection{***}


\subsubsection{***}


\paragraph{***}

% do not change font sizes (except perhaps in the \textbf{References}
% section; see below).


\subsubsection*{Acknowledgments}

% Do not include acknowledgments in the anonymized submission, only in the
% final paper.

\section*{References}

blabla

% References follow the acknowledgments. Use unnumbered first-level
% heading for the references. Any choice of citation style is acceptable
% as long as you are consistent. It is permissible to reduce the font
% size to \verb+small+ (9 point) when listing the references. {\bf
%   Remember that you can go over 8 pages as long as the subsequent ones contain
%   \emph{only} cited references.}
% \medskip


\end{document}
