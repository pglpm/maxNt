\documentclass{article}

\usepackage{nips_2018_author_response}

\newcommand*{\firstdraft}{4 November 2015}
\newcommand*{\published}{22 May 2018}

\newcommand*{\pdftitle}{Response to reviewers}

\newcommand*{\headtitle}{\pdftitle} %\newcommand*{\pdfauthor}{P.G.L.  Porta Mana, V. Rostami, E. Torre, Y. Roudi}

\usepackage[T1]{fontenc} 
\input{glyphtounicode} \pdfgentounicode=1
\usepackage[utf8]{inputenx}
\newcommand*{\bmmax}{3} % reduce number of bold fonts, before bm
\newcommand*{\hmmax}{0} % reduce number of heavy fonts, before bm
\usepackage{textcomp}
\usepackage{amsmath}
\usepackage{mathtools}
\setlength{\multlinegap}{0pt}

\usepackage{amssymb}
\usepackage{amsxtra}

\usepackage[british]{babel}\selectlanguage{british}
\newcommand*{\langfrench}{\foreignlanguage{french}}
\newcommand*{\langgerman}{\foreignlanguage{german}}
\newcommand*{\langitalian}{\foreignlanguage{italian}}
\newcommand*{\langswedish}{\foreignlanguage{swedish}}
\newcommand*{\langlatin}{\foreignlanguage{latin}}
\newcommand*{\langnohyph}{\foreignlanguage{nohyphenation}}

\usepackage[autostyle=false,autopunct=false,english=british]{csquotes}
\setquotestyle{british}
\usepackage{bm}



\usepackage[shortlabels,inline]{enumitem}
\SetEnumitemKey{para}{itemindent=\parindent,leftmargin=0pt,listparindent=\parindent,parsep=0pt,itemsep=\topsep,topsep=0pt}
% \begin{asparaenum} = \begin{enumerate}[para]
% \begin{inparaenum} = \begin{enumerate*}
\setlist[enumerate,2]{label=\alph*}
\setlist[enumerate]{leftmargin=1.5em,topsep=-1ex,itemsep=0.5ex}
\setlist[itemize]{leftmargin=1.5em}
\setlist[description]{leftmargin=\parindent}

%% With euler font cursive for Greek letters - the [1] means 100% scaling
\DeclareFontFamily{U}{egreek}{\skewchar\font'177}%
\DeclareFontShape{U}{egreek}{m}{n}{<-6>s*[0.95]eurm5 <6-8>s*[0.95]eurm7 <8->s*[0.95]eurm10}{}%
\DeclareFontShape{U}{egreek}{m}{it}{<->s*[0.95]eurmo10}{}%
\DeclareFontShape{U}{egreek}{b}{n}{<-6>s*[0.95]eurb5 <6-8>s*[0.95]eurb7 <8->s*[0.95]eurb10}{}%
\DeclareFontShape{U}{egreek}{b}{it}{<->s*[0.95]eurbo10}{}%
\DeclareSymbolFont{egreeki}{U}{egreek}{m}{it}%
\SetSymbolFont{egreeki}{bold}{U}{egreek}{b}{it}% from the amsfonts package
\DeclareSymbolFont{egreekr}{U}{egreek}{m}{n}%
\SetSymbolFont{egreekr}{bold}{U}{egreek}{b}{n}% from the amsfonts package
% Take also \sum, \prod, \coprod symbols from Euler fonts
\DeclareFontFamily{U}{egreekx}{\skewchar\font'177}
\DeclareFontShape{U}{egreekx}{m}{n}{%
       <-7.5>s*[0.9]euex7%
    <7.5-8.5>s*[0.9]euex8%
    <8.5-9.5>s*[0.9]euex9%
    <9.5->s*[0.9]euex10%
}{}
\DeclareSymbolFont{egreekx}{U}{egreekx}{m}{n}
\DeclareMathSymbol{\sumop}{\mathop}{egreekx}{"50}
\DeclareMathSymbol{\prodop}{\mathop}{egreekx}{"51}
\DeclareMathSymbol{\coprodop}{\mathop}{egreekx}{"60}
\makeatletter
\def\sum{\DOTSI\sumop\slimits@}
\def\prod{\DOTSI\prodop\slimits@}
\def\coprod{\DOTSI\coprodop\slimits@}
\makeatother

\usepackage{mathdots}
\usepackage{microtype}

\iffalse
\usepackage[backend=biber,mcite,subentry,citestyle=numeric-comp,bibstyle=numericbringhurst,autopunct=false,sorting=none,sortcites=false,natbib=false,maxnames=8,minnames=8,giveninits=true,block=space,hyperref=true,defernumbers=false,useprefix=true,language=british]{biblatex}
\renewcommand*{\finalnamedelim}{, }
\setcounter{biburlnumpenalty}{1}
\setcounter{biburlucpenalty}{0}
\setcounter{biburllcpenalty}{1}
\DeclareDelimFormat{multicitedelim}{\addsemicolon\space}
\DeclareDelimFormat{postnotedelim}{\space}
\addbibresource{portamanabib.bib}%comment for arxiv
\renewcommand{\bibfont}{\footnotesize}
%\defbibheading{bibliography}[\bibname]{\section*{#1}\addcontentsline{toc}{section}{#1}%\markboth{#1}{#1}
%}
\newcommand*{\citep}{\parencites}
\newcommand*{\citey}{\parencites*}
\renewcommand*{\cite}{\citep}
\providecommand{\href}[2]{#2}
\providecommand{\eprint}[2]{\texttt{\href{#1}{#2}}}
\newcommand*{\amp}{\&}

\newcommand*{\arxiveprint}[1]{%\global\def\arxivp{\arxivsi}%\citeauthor{0arxivcite}\addtocategory{ifarchcit}{0arxivcite}%eprint
\texttt{\urlalt{https://arxiv.org/abs/#1}{arXiv:\hspace{0pt}#1}}%
%\texttt{\href{http://arxiv.org/abs/#1}{\protect\url{arXiv:#1}}}%
%\renewcommand{\arxivnote}{\texttt{arXiv} eprints available at \url{http://arxiv.org/}.}
}
\newcommand*{\mparceprint}[1]{%\global\def\mparcp{\mparcsi}%\citeauthor{0mparccite}\addtocategory{ifarchcit}{0mparccite}%eprint
\texttt{\urlalt{http://www.ma.utexas.edu/mp_arc-bin/mpa?yn=#1}{mp\_arc:\hspace{0pt}#1}}%
%\texttt{\href{http://www.ma.utexas.edu/mp_arc-bin/mpa?yn=#1}{\protect\url{mp_arc:#1}}}%
%\providecommand{\mparcnote}{\texttt{mp_arc} eprints available at \url{http://www.ma.utexas.edu/mp_arc/}.}
}
\newcommand*{\philscieprint}[1]{%\global\def\philscip{\philscisi}%\citeauthor{0philscicite}\addtocategory{ifarchcit}{0philscicite}%eprint
\texttt{\urlalt{http://philsci-archive.pitt.edu/archive/#1}{PhilSci:\hspace{0pt}#1}}%
%\texttt{\href{http://philsci-archive.pitt.edu/archive/#1}{\protect\url{PhilSci:#1}}}%
%\providecommand{\mparcnote}{\texttt{philsci} eprints available at \url{http://philsci-archive.pitt.edu/}.}
}
\newcommand*{\biorxiveprint}[1]{%\global\def\biorxivp{\biorxivsi}%\citeauthor{0arxivcite}\addtocategory{ifarchcit}{0arxivcite}%eprint
\texttt{\urlalt{http://biorxiv.org/content/early/#1}{bioRxiv:\hspace{0pt}#1}}%
%\texttt{\href{http://arxiv.org/abs/#1}{\protect\url{arXiv:#1}}}%
%\renewcommand{\arxivnote}{\texttt{arXiv} eprints available at \url{http://arxiv.org/}.}
}
\newcommand*{\osfeprint}[1]{%
\texttt{\urlalt{https://doi.org/10.17605/osf.io/#1}{doi:10.17605/osf.io/#1}}%
}
\fi

\usepackage{graphicx}

\PassOptionsToPackage{hyphens}{url}\usepackage[hypertexnames=false]{hyperref}
%\usepackage[depth=4]{bookmark}
\hypersetup{colorlinks=true,bookmarksnumbered,pdfborder={0 0 0.25},citebordercolor={0.2 0.1333 0.5333},%bluish
citecolor=mybluishpurple,linkbordercolor={0.0667 0.4667 0.2},%greenish
linkcolor=mypurplishred,urlbordercolor={0.5333 0.1333 0.3333},%reddish
urlcolor=mygreen,breaklinks=true,pdftitle={\pdftitle}}
% \usepackage[vertfit=local]{breakurl}% only for arXiv
\providecommand*{\urlalt}{\href}

\selectlanguage{british}\frenchspacing
% \usepackage{booktabs}       % professional-quality tables
% \usepackage{amsfonts}       % blackboard math symbols
% \usepackage{nicefrac}       % compact symbols for 1/2, etc.
%\usepackage{lipsum}

\begin{document}
We thank the reviewers for their analysis, appreciative words, and
suggestions. Owing to space constraints our reply below does not address
the points that can be easily addressed by amending our manuscript.

\bigskip

\textbf{Reviewer 1}
\begin{enumerate}[wide]
\item \emph{\enquote{$P(s)$}:} Unfortunately the Reviewer's comment is
  truncated; we aren't fully sure if we are interpreting \enquote{$P(s)$}
  correctly. We assume it to be the empirical frequency distribution of the
  sample activity, equivalent to $n$ moment constraints. This constraint is
  one of those used in step 2. of the protocol (lines 27--56), to be
  compared with the other constraints. We may add that using the
  empirical frequencies as constraints at the sample level is questionable:
  activities with zero frequencies lead to zeroes in the maximum-entropy
  distribution, and these are unreasonable because the sample size is much
  smaller than the state-space size. This indicates that maximum-entropy is
  used beyond its range of validity. The approach at the population level +
  marginalization cures this problem.
% Its use, however, addresses a different question than
%   the one discussed in our introduction, namely to define and measure
%   \enquote{cooperativity} by comparing maximum-entropy distributions
%   constructed from different numbers of moments.
\item \emph{Puzzling finds in the literature:} We apologize for having
  forgotten to discuss this; this oversight can be easily fixed.
\end{enumerate}

\bigskip

\textbf{Reviewer 2}
\begin{enumerate}[wide]
\item \emph{Comparison with other frameworks and long intro:} We
  respectfully acknowledge the Reviewer's personal stand on the
  \enquote{results vs introduction} point. Our opinion is that at times it
  is important to focus on the problem and the principles of the method
  purported to solve it, leaving more detailed results till later. The
  reason is this: if the method is misapplied, the results are actually
  irrelevant or even misleading -- and yet they may \emph{appear}
  interesting, risking to steal our focus from the principles. What's worse
  is that the results themselves may not reveal, or may even conceal, that
  the method was misapplied. We think that this has happened in many uses
  of the maximum-entropy method in neuroscience.

  The modicum of results we presented is meant to bring to light this
  possible misapplication. We called this a \enquote{dilemma} because we do
  not wish to advertise the newly derived equations as the final solution
  to the problem. At this stage we consider a careful description of the
  problem critical, and a comparison with more advanced methods still
  premature.

  Also, we would like to reach a broader audience than the one specifically
  interested in \enquote{cooperativity}. Our primary goal is to raise
  awareness about this possible pitfall of the maximum-entropy method, and
  to show a possible solution. We fear that focusing on further results
  specific to cooperativity would conceal the wider bearing of our message.


%   As this Reviewer and Reviewer 2 remark, the points and findings above
% deserve more discussion. Unfortunately there are space constraints, and we
% preferred to give more room to a careful analysis of the principles behind
% the method, to see whether the method is correctly applied. Our main wish
% is that our readers will give more thoughts to these matters. We could
% hurry up and produce a lot of mathematical results; but, whether positive
% or negative, any results are meaningless if the method isn't meaningfully
% applied. And it isn't always the case that a meaningful application of a
% method can be judged by its results alone.
  % But we believe that it's also important to carefully
  % examine (1) the question we're asking, and (2) which first principles can
  % be used to translate it into a mathematical problem. Some literature
  % develops very refined -- though often ad hoc -- mathematical techniques,
  % but then leaves us unsatisfied, with the lingering question \enquote{why
  %   were we doing this?}. This is the reason for our long introductory
  % discussion about the problem and about the first principles that can be
  % used to address it. See also our reply to Reviewer 3 below.
\item \emph{Effect of firing rates and correlations:} The empirical
  correlations determine (rather than affect) the result; see the protocol
  1--3, lines 27--56. Firing rates surely affect the results, but they do
  so through the intermediary of the correlations. The question about the
  effect of firing rates actually brings us back to an analysis of the
  initial problem: do we want a measure of \enquote{cooperativity} that is
  invariant under firing-rate changes? do the principles behind the
  maximum-entropy method provide this? from this point of view, are we
  misapplying the method?
  
  % We wonder whether maximum-entropy can provide this. Again, this is a
  % question that should be addressed in defining the problem and the
  % principles behind the method to solve it.
%   likely affects the results and the
%   quantification of \enquote{cooperativity}. The question is: Is this an
%   issue? is our idea of \enquote{cooperativity} invariant with respect to
%   the firing rate, or not? Requirements of this kind constrain the
%   translation of cooperativity into a mathematical quantity. This again
%   shows the importance of defining the question we're asking first.
% \item Regarding correlations: the method quantifies cooperativity using the
%   measured correlations; see the \enquote{protocol} 1--3 in the
%   Introduction. Thus, correlations determine the result rather than
%   affecting it.
\item \emph{Bayesian approaches and sampling:} The literature indeed offers
  fully Bayesian approaches, which we also prefer. But maximum-entropy
  methods still abound in the neuroscientific literature, and maybe they
  can be reasonably motivated. Our goal is to point out the subtlety in
  applying these methods. The literature we have explored makes very little
  use of the basic sampling formulae presented in our paper. This is
  surprising: such formulae are surely essential even in a very basic
  analysis of the data.
\end{enumerate}

\bigskip

\textbf{Reviewer 3}
\begin{enumerate}[wide]
\item \emph{Whether it matters in practice:} We agree with the Reviewer
  that pointing out the circumstances where the difference matters would be
  useful. For instance, the thinner
  right tail of the population-level distribution observed by the Reviewer\\
  \begin{minipage}[t]{0.74\linewidth}indicates that higher-order
    correlations observed in the data are important (this is the question
    Schneidman et al. were addressing in their research, which popularized
    maximum-entropy in neuroscience). Note that a more entropic
    distribution is not necessarily more correct (after all, we could just
    use a uniform distribution otherwise). Application of the method with
    higher-order constraints can show more interesting differences between
    sample- and population-level distributions, e.g. a bimodality not
    observed in the empirical frequencies, as in the plot on the right
    (bimodal distribution = population-level). This also leads to different
    conclusion about the relative importance of the correlations.
  \end{minipage}
  \hspace{\fill}
    \smash{\begin{minipage}[t]{0.24\linewidth}\vspace{0pt}
      \includegraphics[width=\linewidth]{zoom_4mom.png}
    \end{minipage}}
%     % \setlength{\intextsep}{0.5ex}% with wrapfigure
%     \begin{figure}[h!]%{r}{0.4\linewidth} % with wrapfigure
%   \centering
% %\caption{***}\label{fig:comparison_a5}
% \end{figure}

  But we believe that \emph{even if the differences were negligible, the
    results of the paper would still matter}: the point is that we could
  not have known about the presence or absence of differences, if we had
  not faced the whole problem and derived a formula showing that the
  difference were negligible.
\item\emph{Unnecessarily principled:} Regarding our motivation for emphasizing
  principles, we kindly ask the Reviewer to read our reply 1. to Reviewer~2
  above. Possibly the writing could be improved on our part to make the
  paper flow more smoothly, without sacrificing the more philosophical
  discussion?
  % We agree with the Reviewer that the main formulae of the paper, (17)
  % and (19), probably have little \emph{experimental} use today. But the
  % formulae and the analysis behind them are important exactly for the
  % points raised by the Reviewer:
% \item Calculations show that the population-level distribution can be less
%   or more entropic than the sample-level one. But a more entropic
%   distribution is not necessarily more correct (we could just use a uniform
%   distribution then). The question is how the constraints are appropriately
%   applied in the method. 
% \item As this Reviewer and Reviewer 2 remark, the points and findings above
% deserve more discussion. Unfortunately there are space constraints, and we
% preferred to give more room to a careful analysis of the principles behind
% the method, to see whether the method is correctly applied. Our main wish
% is that our readers will give more thoughts to these matters. We could
% hurry up and produce a lot of mathematical results; but, whether positive
% or negative, any results are meaningless if the method isn't meaningfully
% applied. And it isn't always the case that a meaningful application of a
% method can be judged by its results alone.
% \item We agree with Reviewer 3 that the paper emphasizes some philosophical
%   aspects; intentionally so. Sadly we can't make all kinds of readers
%   happy; a choice of audience is necessary. As readers ourselves we
%   appreciate when a paper begins by asking: \enquote{What is the question?
%     is it possible to translate it into a mathematical problem? which
%     principles can we use to make such translation?}. There are papers that
%   develop very refined mathematical techniques but leave us unsatisfied,
%   with the lingering question \enquote{why are we doing this?}. This is the
%   reason why we try to emphasize these kinds of questions. But we are sure
%   that part of the NIPS audience will appreciate this emphasis.
\end{enumerate}
\end{document}
